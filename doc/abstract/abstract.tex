\documentclass{article}

\usepackage{xcolor}
\newcommand\mycomments[1]{\textcolor{red}{#1}}


\title{Reducing variability and turbines, increasing output: the potential for repowering US wind turbines}

\author{\small{Peter Regner$^{1}$, Katharina Gruber$^{1}$, Johannes Schmidt$^{1}$, Claude Kl\"ockl$^{1}$}}
\date{$^{1}$ \small{\textit{Institute for Sustainable Economic Development,
University of Natural Resources and Life Sciences Vienna (BOKU)}}}


\begin{document}

\maketitle

\small{
Many countries have established significant wind power capacities over the last decades. Still, there is a growing demand for wind power generation to meet climate change mitigation goals. Besides adding renewable generation capacities at new locations, re-powering old locations, i.e. replacing old turbines by new ones, is a way of maximizing energy output at productive locations, re-using existing infrastructure such as roads and transmission grid connections, and reducing conflicts over land.\\

We study the potential for repowering in the United States (US), thus we asses the possible gains from retrofitting the current US wind power fleet with latest technology. For that purpose, we use the ERA5 reanalysis data set as a source of meteorological data for wind power simulation and a comprehensive database of US wind turbines, publically available. We motivate our choice of the ERA5 data set by a comparison with other potential data sources (i.e. MERRA-2) and assess if the global wind power atlas (GWA) is useful in reducing bias in the reanalysis data. We first simulate output of the current fleet and validate it against observed wind power output as provided by the energy information administration.\\

In a second step, we optimize the deployment of new turbines at all locations, taking into account wind turbine technology currently installed, possible replacement turbines, and spacing area necessary for different turbine types. Typically, wind turbine efficiency is described through so-called power curves, that result from benchmarking tests of the turbine producer. While the available turbine database provides relatively complete information about installed turbines and their locations, the data on the technical specifications is limited to a few key figures. In some cases, the turbine model is not available or manufacturers do not provide information on power curves for installed turbines. Therefore we opt to employ a model that infers turbine output satisfyingly from as little information as possible. \\

The consequences of repowering are discussed in terms of level and variance of power output, costs, number of turbines, and swept rotor area. Very preliminary results show that output can be increased substantially, even if the number of turbines is decreased. This is mainly a result of a larger area swept by the rotors, but also a consequence of improved technology.
}
\end{document}
